%%%%%%%%%%%%%%%%%%%%%%%%%%%%%%%%%%%%%%%%%%%%%%%%%%%%%%%%%%%%%%%%%%%%%%
% LaTeX Template: Curriculum Vitae Mejorado y Corregido
%
% Basado en la plantilla original de:
% http://www.howtotex.com/
%
% Modificado para el CV de Wilson Eduardo Jerez Hernandez.
%%%%%%%%%%%%%%%%%%%%%%%%%%%%%%%%%%%%%%%%%%%%%%%%%%%%%%%%%%%%%%%%%%%%%%
\documentclass[paper=a4,fontsize=11pt]{scrartcl} % KOMA-article class

\usepackage[spanish]{babel}
\usepackage[utf8]{inputenc}
\usepackage[T1]{fontenc}
\usepackage[protrusion=true,expansion=true]{microtype}
\usepackage{amsmath,amsfonts,amsthm}     % Paquetes de matemáticas
\usepackage{graphicx}                    % Para incluir imágenes
\usepackage[svgnames]{xcolor}            % Colores por nombre 'svgnames'
\usepackage{geometry}                    % Control de márgenes
  \geometry{left=2cm,right=2cm,top=2.5cm,bottom=2.5cm}
\usepackage{url}                         % Para enlaces
\usepackage{enumitem}                    % Para listas personalizadas
\usepackage{hyperref}                    % Para hipervínculos

\frenchspacing              % Mejor espaciado después de puntos
\pagestyle{empty}           % Sin número de página / encabezados / pies

%%% Configuración de secciones usando KOMA-Script
\setkomafont{section}{\usefont{OT1}{phv}{b}{n}\color{MidnightBlue}\Large}
\setkomafont{sectionentry}{\color{MidnightBlue}}
\setkomafont{descriptionlabel}{\itshape}

%%% Definición de longitudes y comandos personalizados
\newlength{\spacebox}
\settowidth{\spacebox}{8888888888} % Para alinear texto
\newcommand{\sepspace}{\vspace*{0.8em}}

\newcommand{\MyName}[1]{ % Nombre grande
  {\Huge \usefont{OT1}{phv}{b}{n} #1}%
  \par \normalsize \normalfont}

\newcommand{\MySlogan}[1]{ % Eslogan o subtítulo
  {\large \usefont{OT1}{phv}{m}{n}\textit{#1}}%
  \par \normalsize \normalfont}

\newcommand{\NewPart}[1]{\section*{\uppercase{#1}}}

\newcommand{\PersonalEntry}[2]{%
  \noindent\hangindent=2em\hangafter=0 % Sangría
  \parbox{\spacebox}{\textit{#1}}%
  \hspace{1.5em} #2 \par}

\newcommand{\SkillsEntry}[2]{%
  \noindent\hangindent=2em\hangafter=0
  \parbox{\spacebox}{\textit{#1}}%
  \hspace{1.5em} #2 \par}

\newcommand{\EducationEntry}[4]{%
  \noindent \textbf{#1} \hfill \textbf{#2} \par
  \noindent \textit{#3} \par
  \noindent \small #4 
  \normalsize \par \sepspace
}

\newcommand{\WorkEntry}[4]{%
  \noindent \textbf{#1} \hfill \textbf{#2} \par
  \noindent \textit{#3} \par
  \noindent \small #4 
  \normalsize \par \sepspace
}

%%%%%%%%%%%%%%%%%%%%%%%%%%%%%%%%%%%%%%%%%%%%%%%%%%%%%%%%%%%%%%%%%%%%%%
%%% Inicio del Documento
%%%%%%%%%%%%%%%%%%%%%%%%%%%%%%%%%%%%%%%%%%%%%%%%%%%%%%%%%%%%%%%%%%%%%%
\begin{document}

%%% ------------------------------------------------------------------
%%% ENCABEZADO
%%% ------------------------------------------------------------------
\MyName{Wilson Eduardo Jerez Hernández}
\MySlogan{B.Sc. in Mathematics | Data Scientist | Financial Analyst}

\sepspace

%%% ------------------------------------------------------------------
%%% DETALLES DE CONTACTO
%%% ------------------------------------------------------------------
\NewPart{Detalles Personales}
\PersonalEntry{Nacimiento}{11 de Mayo, 2001}\par
\PersonalEntry{Dirección}{CL 75 \textbackslash 80B-20, Bogotá DC, Colombia}\par
\PersonalEntry{Teléfono}{+57 300-392-6817}\par
\PersonalEntry{GitHub}{\href{https://github.com/wilwil186}{github.com/wilwil186}}\par
\PersonalEntry{Correo}{\href{mailto:edujerezwilson@gmail.com}{edujerezwilson@gmail.com}}\par
\PersonalEntry{LinkedIn}{\href{https://www.linkedin.com/in/wilson-eduardo-jerez-hern\%C3\%A1ndez-833858206/}{linkedin.com/in/wilson-eduardo-jerez-hernández-833858206}}\par


\sepspace

%%% ------------------------------------------------------------------
%%% PERFIL PROFESIONAL
%%% ------------------------------------------------------------------
\NewPart{Perfil Profesional}
Soy un apasionado de la tecnología y la ciencia de datos, con una sólida formación en matemáticas y experiencia en análisis de datos, modelado y aprendizaje automático. Poseo habilidades en programación y manejo de bases de datos, lo que me permite desarrollar soluciones innovadoras para resolver problemas complejos. Mi enfoque está en la mejora continua y en la aplicación de las últimas tecnologías para aportar valor en proyectos desafiantes. Disfruto del trabajo en equipo y de colaborar en entornos multidisciplinarios para alcanzar objetivos comunes.

\sepspace

%%% ------------------------------------------------------------------
%%% EXPERIENCIA
%%% ------------------------------------------------------------------
\NewPart{Experiencia}

\WorkEntry{Científico de Datos}{jul. 2024 - ene. 2025}{BBVA, Bogotá, Colombia}{%
\begin{itemize}[leftmargin=*, noitemsep]
    \item Implementé técnicas de Procesamiento de Lenguaje Natural (NLP) para extraer información valiosa de datos no estructurados.
    \item Desarrollé scripts de automatización, optimizando procesos de extracción y limpieza de datos.
    \item Analicé grandes volúmenes de datos utilizando Apache Spark, construyendo modelos de aprendizaje automático para predecir tendencias financieras.
    \item Diseñé dashboards interactivos con herramientas como Tableau y Power BI para visualizar datos y comunicar hallazgos a stakeholders.
    \item Colaboré en equipos multidisciplinarios, integrando análisis de datos en estrategias de negocio para mejorar la toma de decisiones.
\end{itemize}
}

\WorkEntry{Líder Semillero Iprea}{ago. 2022 - jul. 2024}{Universidad Distrital Francisco José de Caldas, Bogotá, Colombia}{%
\begin{itemize}[leftmargin=*, noitemsep]
    \item Dirigí el Semillero Iprea, coordinando actividades de investigación y proyectos aplicados en estadística.
    \item Guié a los miembros en la aplicación de conceptos estadísticos, desde el análisis de datos hasta la implementación de modelos predictivos.
    \item Organicé y facilité sesiones de formación y talleres para fortalecer las habilidades técnicas y de trabajo en equipo de los integrantes.
    \item Lideré proyectos que aplicaron técnicas estadísticas para resolver problemas reales, promoviendo la innovación y la colaboración.
\end{itemize}
}

\WorkEntry{Proyectos en GitHub}{2020 - Actual}{Perfil Personal de GitHub}{%
\begin{itemize}[leftmargin=*, noitemsep]
    \item Desarrollo de proyectos en Python, R y Go enfocados en análisis de datos, machine learning y desarrollo de software.
    \item Utilización de Git y GitHub para control de versiones y colaboración en proyectos de código abierto.
    \item Implementación de algoritmos de aprendizaje automático para la predicción y clasificación de datos.
    \item Creación de visualizaciones interactivas y dashboards para la presentación de resultados.
    \item Documentación detallada de cada proyecto para facilitar la comprensión y reutilización por parte de otros desarrolladores.
\end{itemize}
}

\sepspace

%%% ------------------------------------------------------------------
%%% EDUCACIÓN
%%% ------------------------------------------------------------------
\NewPart{Educación}

\EducationEntry{B.Sc. in Mathematics}{2019 - 2024}{Universidad Distrital Francisco José de Caldas}{%
Profundización en cálculo, álgebra, estadística y geometría, aplicando estos conocimientos en la resolución de problemas y modelización matemática. Participación en proyectos de investigación y desarrollo de modelos matemáticos para aplicaciones prácticas.
}

\sepspace

%%% ------------------------------------------------------------------
%%% LICENCIAS Y CERTIFICACIONES
%%% ------------------------------------------------------------------
\NewPart{Licencias y Certificaciones}
\begin{itemize}[leftmargin=*, noitemsep]
  \item \textbf{Financial Analyst} \\
        \textit{Platzi} \\
        Expedición: Ene. 2025 

  \item \textbf{Herramientas de AI para Programadores} \\
        \textit{Platzi} \\
        Expedición: Dic. 2024 

  \item \textbf{Introducción a AWS: Conceptos de la nube} \\
        \textit{LinkedIn} \\
        Expedición: Nov. 2024 \\
        \textbf{Aptitudes:} Amazon Web Services (AWS) 

  \item \textbf{AWS Cloud Practitioner Essentials} \\
        \textit{Amazon Web Services (AWS)} \\
        Expedición: Oct. 2024 

  \item \textbf{Natural Language Processing with Classification and Vector Spaces} \\
        \textit{DeepLearning.AI} \\
        Expedición: Oct. 2024 

  \item \textbf{Prompt Engineering: Aprende a hablar con una inteligencia artificial generativa} \\
        \textit{LinkedIn} \\
        Expedición: Oct. 2024 \\
        \textbf{Aptitudes:} Prompt Engineering 

  \item \textbf{Sistemas GNU Linux} \\
        \textit{Universidad Distrital Francisco José de Caldas} \\
        Expedición: Sept. 2024 

  \item \textbf{Curso de Portugués para Hispanohablantes} \\
        \textit{Platzi} \\
        Expedición: Sept. 2023 

  \item \textbf{Participación en el Primer Congreso Caldas de Matemáticas – CACOMA} \\
        \textit{Universidad de Caldas} \\
        Expedición: Ago. 2023

  \item \textbf{Certificado XXIII CCM UPTC - Tunja} \\
        \textit{Sociedad Colombiana de Matemáticas} \\
        Expedición: Jun. 2023

  \item \textbf{Escuela de Blockchain y Criptomonedas} \\
        \textit{Platzi} \\
        Expedición: Jul. 2021 

  \item \textbf{Ciencias} \\
        \textit{Platzi} \\
        Expedición: Mar. 2021 
\end{itemize}

\sepspace

%%% ------------------------------------------------------------------
%%% HABILIDADES
%%% ------------------------------------------------------------------
\NewPart{Habilidades}

\begin{itemize}[leftmargin=*, noitemsep]
    \item \textbf{Lenguajes de Programación:} Python, R, Go.
    \item \textbf{Herramientas y Tecnologías:} Git, GitHub, Apache Spark, Tableau, Power BI.
    \item \textbf{Análisis de Datos:} Estadística, Machine Learning, Deep Learning, Análisis Exploratorio de Datos, Visualización de Datos.
    \item \textbf{Ofimática:} Microsoft Office, Google Suite, \LaTeX.
    \item \textbf{Idiomas:} Español (Nativo), Inglés (Intermedio), Portugués (Intermedio).
\end{itemize}

\sepspace

%%% ------------------------------------------------------------------
%%% EVENTOS ACADÉMICOS
%%% ------------------------------------------------------------------
\NewPart{Eventos Académicos}

\EducationEntry{XXIII CCM UPTC - Tunja}{2023}{Asistente}{%
Participación como asistente en el XXIII Congreso Colombiano de Matemáticas organizado por la UPTC en Tunja.
}

\EducationEntry{Primer Congreso Caldas de Matemáticas (CACOMA)}{2023}{Ponente}{%
Ofrecí una conferencia titulada \textit{Predicción de criptomonedas con Random Forest} en la Ciudad de Manizales, contribuyendo al fortalecimiento del estudio de las matemáticas en la región.
}

\EducationEntry{I Semana de Ciencias, Universidad Distrital}{2022}{Cursillista}{%
Participación en el curso \textit{Una introducción al Aprendizaje de Máquina Asistido con R y Python} durante la I Semana de Ciencias de la Universidad Distrital.
}

\sepspace

\end{document}

